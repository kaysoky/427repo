\documentclass[a4paper, 12pt]{report}

\usepackage[margin=1in]{geometry}
\usepackage{amsmath}
\usepackage{amssymb}
\usepackage{listings}
\usepackage{graphicx}

\begin{document}
    \begin{center}
        {\LARGE Project One} \\
        CSE 427 A \\
        Joseph Wu \makebox[0pt][l]{{\tiny (0911680)}} \\
        January 20-28, 2014 \\
        {\tiny Typeset via \LaTeX}
    \end{center}
    
\section{Build instructions}
    The algorithm is implemented in ``align.py''.  
    It is called from either the commandline (see ``Compare.bat'') 
        or from the Python shell (see ``compare\_proteins.py'').
    All files containing results can be generated by running ``Compare.bat''.
    Building this PDF also depends on those files.
    
    Note: the Numpy Python module is required.
    
\section{Test Case}
    The results for comparing ``deadly'' to ``ddgearlyk'' can be found in (``AB.out'')
    and are included in this write-up for convenience:
    \lstinputlisting[basicstyle=\ttfamily]{AB.out}
    
\section{P15172 versus Q10574 and O95363}
    For the sake of modularity, the empirical probability 
        calculation is done separately from the general protein comparison.
        Hence, the results do overlap somewhat.
    \lstinputlisting[basicstyle=\scriptsize\ttfamily]{Empirical_P15172_Q10574.out}
    \lstinputlisting[basicstyle=\scriptsize\ttfamily]{Empirical_P15172_O95363.out}
    
\section{Protein comparison}
    The results for each protein pair comparison can be found in ``Proteins.out''.
    This is the summary of pair-wise alignment scores:
    
    \include{Proteins}
    
    The raw output is included in this write-up for convenience:
    \lstinputlisting[basicstyle=\scriptsize\ttfamily]{Proteins.out}
\end{document}