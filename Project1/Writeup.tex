\documentclass[a4paper, 12pt]{report}

\usepackage[margin=1in]{geometry}
\usepackage{amsmath}
\usepackage{amssymb}
\usepackage{listings}
\usepackage{graphicx}
\usepackage{xcolor}

\begin{document}
    \begin{center}
        {\LARGE Project One} \\
        CSE 427 A \\
        Joseph Wu \makebox[0pt][l]{{\tiny (0911680)}} \\
        January 20-28, 2014 \\
        {\tiny Typeset via \LaTeX}
    \end{center}
    
\section{Build instructions}
    The algorithm is implemented in ``align.py''.  
    It is called from either the commandline (see ``Compare.bat'') 
        or from the Python shell (see ``compare\_proteins.py'').
    All files containing results can be generated by running ``Compare.bat''.
    Building this PDF also depends on those files.
    
    Note: the Numpy Python module is required.
    
\section{Test Case}
    The results for comparing ``deadly'' to ``ddgearlyk'' can be found in (``AB.out'').
    Score matrix:
    \begin{align*}
        \left[ \begin{array}{*{11}{c}}
            - & - & d & d & g &  e &  a & r &  l &  y &  k \\
            - & 0 & 0 & 0 & 0 &  0 &  0 & 0 &  0 &  0 &  0 \\
            d & 0 & 6 & \color{red} 6 & 2 &  2 &  0 & 0 &  0 &  0 &  0 \\
            e & 0 & 2 & 8 & \color{red} 4 &  \color{red} 7 &  3 & 0 &  0 &  0 &  1 \\
            a & 0 & 0 & 4 & 8 &  4 & \color{red} 11 & 7 &  3 &  0 &  0 \\
            d & 0 & 6 & 6 & 4 & 10 &  7 & \color{red} 9 &  5 &  1 &  0 \\
            l & 0 & 2 & 2 & 2 &  6 &  9 & 5 & \color{red} 13 &  9 &  5 \\
            y & 0 & 0 & 0 & 0 &  2 &  5 & 7 &  9 & \color{red} 20 & 16 
        \end{array} \right]
    \end{align*}
    
    \lstinputlisting[basicstyle=\ttfamily]{AB.out}
    The p-value is calculated with 100 random permutations.  
    
\section{P15172 versus Q10574 and O95363}
    For the sake of modularity, the empirical probability 
        calculation is done separately from the general protein comparison.
        Hence, the results do overlap somewhat.
        
    See ``Empirical\_P15172\_Q10574.out'' and ``Empirical\_P15172\_O95363.out'' for the full set of results.
    In the case of P15172 and Q10574, the p-value is less than 0.0005, based on 2000 trials.  
    In the case of P15172 and O95363, a better alignment occurs about half the time.  
    Also based on 2000 trials.
    
\section{Protein comparison}
    The results for each protein pair comparison can be found in ``Proteins.out''.
    This is the summary of pair-wise alignment scores:
    
    \input{Proteins}
    
    The raw output is included in this write-up for convenience:
    \lstinputlisting[basicstyle=\scriptsize\ttfamily]{Proteins.out}
\end{document}