\documentclass[a4paper, 12pt]{report}

\usepackage[margin=1in]{geometry}
\usepackage{amsmath}
\usepackage{amssymb}
\usepackage{listings}

\begin{document}
    \begin{center}
        {\LARGE Project Three} \\
        CSE 427 A \\
        Joseph Wu \makebox[0pt][l]{{\tiny (0911680)}} \\
        February 11-20, 2014 \\
        {\tiny Typeset via \LaTeX}
    \end{center}

\section{Build Instructions}
    A single iteration of the program is run with: \\
    \texttt{python viterbi.py TestViterbi.fna --inE Emission00.json --inT Transition00.json --outE Emission01.json --outT Transition01.json} \\

    \noindent Multiple iterations are comprised of using the output from the previous call: \\
    \texttt{python viterbi.py TestViterbi.fna --inE Emission01.json --inT Transition01.json --outE Emission02.json --outT Transition02.json} \\
    \texttt{python viterbi.py TestViterbi.fna --inE Emission02.json --inT Transition02.json --outE Emission03.json --outT Transition03.json} \\
    \texttt{...} \\

    \noindent The script used to run the 10 iterations is included as ``Iterate10.bat''.

\section{(Super slow) Computer Specs}
    1.3 GHz dual-core (my implementation is single threaded) \\
    800 MHz front side bus \\
    4 GB RAM \\
    Standard hard disk (not solid state)

\section{Format of iterations}
    For all iterations, I have included the parameters in JSON format. \\
    The emission probability matrix contains one upper level key to represent the state
        and a lower level key to represent the emitted letter. \\
    The transition probability matrix contains one upper level key to represent the starting state
        and a lower level key to present to the following state.  \\
    The low-GC state is state ``0''
        and the high-GC state is state ``1''.

% Imports the a single iteration of input/output
\newcommand{\IncludeOutput}[1]{
    \subsection{Iteration #1}
    Emission Probabilities:
    \lstinputlisting[basicstyle=\small\ttfamily]{Emission#1.json}
    Transition Probabilities:
    \lstinputlisting[basicstyle=\small\ttfamily]{Transition#1.json}
    Viterbi output:
    \lstinputlisting[basicstyle=\small\ttfamily]{Results#1.out}
}
\section{Iterations}
    \IncludeOutput{00}
    \IncludeOutput{01}
    \IncludeOutput{02}
    \IncludeOutput{03}
    \IncludeOutput{04}
    \IncludeOutput{05}
    \IncludeOutput{06}
    
    \textbf{Note: At this point, the probabilities have converged}
    
    \IncludeOutput{07}
    \IncludeOutput{08}
    \IncludeOutput{09}

\section{First 10 Hits}
    \begin{enumerate}
        \item  97326\ldots{}97542  \\
            Includes the complement of the coding sequence for tRNA-Met-1
        \item  97627\ldots{}97824  \\
            Includes the coding sequence for tRNA-Leu-1
        \item 111764\ldots{}111857 \\
            Includes the coding sequence for tRNA-SeC-1
        \item 118079\ldots{}118180 \\
            Lies in the region preceeding another gene eIF-2BB, 
            which catalyses the binding of GTP to IF2
        \item 138345\ldots{}138420 \\
            Includes the complement of the coding sequence for tRNA-Val-1
        \item 154610\ldots{}157698 \\
            Includes the complement of the coding sequence for MJrrnA23S (23S ribosomal RNA).  
            This is a subcomponent of a subunit of the prokaryotic ribosome
        \item 157782\ldots{}159592 \\
            Includes the complement of the coding sequence for both tRNA-Ala-1 and MJrrnA16S (16S ribosomal RNA)
        \item 186974\ldots{}187068 \\
            Includes the coding sequence for tRNA-Ser-1
        \item 190831\ldots{}190908 \\
            Includes the coding sequence for tRNA-Pro-1
        \item 215200\ldots{}215297 \\
            Includes the coding sequence for tRNA-Leu-1
    \end{enumerate}

\section{Comments}
    Since I was programming on a relatively slow computer (see computer specs),
        I tested my algorithm with a much smaller file (1500 bases long).
    This file was made by taking 1000 bases from the ``NC\_000909.fna'' file
        and inserting 500 bases worth of high GC content in the middle.
    (Comments are also included in the test file.)
    Originally, the middle section only had G's and C's,
        but since that resulted in a 0 probability of A's and T's,
        it wouldn't fit into log-space.
    I had to pad the middle with a few A's and T's to proceed.

    In terms of performance enhancements, since I'm writing in Python,
        it will no doubt be slower than a C implementation.
    In the slowest portion, running the Viterbi algorithm,
        I at first used O(n) for-loops and after confirming they worked,
        I changed them into Python-reduce() functions.
    That hopefully reduces the amount of time spent in the interpreter.
\end{document}