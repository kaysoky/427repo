\documentclass[a4paper, 12pt]{report}

\usepackage[margin=1in]{geometry}
\usepackage{amsmath}
\usepackage{amssymb}
\usepackage{listings}
\usepackage{pgfplots}
\pgfplotsset{compat=1.7}

\begin{document}
    \begin{center}
        {\LARGE Project Four} \\
        CSE 427 A \\
        Joseph Wu \makebox[0pt][l]{{\tiny (0911680)}} \\
        March 6-13, 2014 \\
        {\tiny Typeset via \LaTeX}
    \end{center}

\section{(Super slow) Computer Specs}
    1.3 GHz dual-core (single threaded program) \\
    800 MHz front side bus \\
    4 GB RAM \\
    Standard hard disk (Average latency 5.6 ms, Random seek time 14 ms)

\section{Candidate Selection}
    \subsection{Usage intructions}
    For candidate selection, I wrote a script with various parameters to experiment with.
    The tool is self-documented and I included the commands used in the following sections: \\
    \texttt{python filter.py -h}
    \lstinputlisting[basicstyle=\scriptsize\ttfamily]{Usage-Filter.out}

    \subsection{``Junk'' filtration}
        Since we are working with files that do not fit in memory,
            I used the ``generator'' pattern in Python to keep memory utilization at a minimum.
        I planned to first parse the SAM file and filter out the most obvious non-candidates:
            near perfect matches (3 or more mismatches)
            and sequences without a significant poly-A tail (at least 3 A's).
        I made the filter for poly-A tails very lenient,
            in that it also counts the letters R, W, M, D, H, V, and N
            as part of the tail.
        These are, as part of IUPAC standard nucleotide notation, possible A's.
        I would then output that result in a JSON file,
            which could be piped back into the same program for further filtering.

        \textbf{Extra credit:} The poly-A tail filter also includes a poly-T head filter for reverse complements.

        The following command was run for this step: \\
        \texttt{python filter.py --min\_mismatch 3 --min\_polyAlen 3 --compute\_background \\
                00-01-background.json --verbose all.sam 00-01-MMM3-MPA3.json}

        With the following output:
        \lstinputlisting[basicstyle=\small\ttfamily]{00-01-MMM3-MPA3.out}
        Even though this filter was very generous, it conserved about 1 in 200 sequences,
            leaving a very manageable number of data points.
        From here, more stringent filters can be applied within a reasonable amount of time.

    \subsection{Background model}
        The majority of the runtime of the preceeding step was consumed by the calculation of the background model.
        If I exclude this process, the runtime decreases to about 40 minutes.
        However, as a result of counting the nucleotide occurrences in this way,
            I'm fairly confident that the background frequencies are slightly biased towards A's and T's.
        This will undoubtedly give a different relative entropy than a uniform background model.

        This is the calculated background model, with probabilities rounded down to 5 decimal places: \\
        \begin{tabular}{r*{6}{|c}}
            Position & 1 & 2 & 3 & 4 & 5 & 6 \\ \hline
            A & 0.28249 & 0.28280 & 0.28269 & 0.28286 & 0.28263 & 0.28151 \\ \hline
            C & 0.23456 & 0.23381 & 0.23410 & 0.23416 & 0.23435 & 0.23502 \\ \hline
            G & 0.21025 & 0.20966 & 0.20938 & 0.20935 & 0.20919 & 0.20984 \\ \hline
            T & 0.27270 & 0.27373 & 0.27382 & 0.27362 & 0.27382 & 0.27363
        \end{tabular}

    \subsection{Normalization and refinement}
        My next step was to reduce the set of candidates to a stronger set.
        I wanted to first modify all reverse complements into non-reversed non-complements
            in order to simplify calculations.  I did this with the following command: \\
        \texttt{python filter.py --dereverse --verbose 00-01-MMM3-MPA3.json 01-02-DRV.json}

        I followed this with a filter for a longer poly-A tail: \\
        \texttt{python filter.py --min\_polyAlen 10 --verbose 01-02-DRV.json 02-03-MPA10.json}

        With the following output:
        \lstinputlisting[basicstyle=\small\ttfamily]{02-03-MPA10.out}

        I compared this result with the same filter on the previous filter result
            in order to check my revere complement code.
        If correct, I should get the same number of results: \\
        \texttt{python filter.py --min\_polyAlen 10 --verbose 00-01-MMM3-MPA3.json 01-02-TEST.json}

        Next I got rid of all the sequences with major mismatching the 3' UTR region
            and all the sequences with short leading sections: \\
        \texttt{python filter.py --max\_non\_tail\_mismatches 4 --min\_UTRlen 18 --verbose \\
                02-03-MPA10.json 03-04-NTM4-MUL18.json}

        With the following output:
        \lstinputlisting[basicstyle=\small\ttfamily]{03-04-NTM4-MUL18.out}
        Peeking at the data, it seems like the remaining data is suitable as a set of candidates.

\section{Script usage instructions}
    \texttt{python scanner.py -h}
    \lstinputlisting[basicstyle=\scriptsize\ttfamily]{Usage-Scanner.out}
    \texttt{python entropy.py -h}
    \lstinputlisting[basicstyle=\scriptsize\ttfamily]{Usage-Entropy.out}
    \texttt{python meme.py -h}
    \lstinputlisting[basicstyle=\scriptsize\ttfamily]{Usage-Meme.out}

    I ran the candidates through the analysis and relative entropy calculations
        with both computed and uniform background models.

\section{Computed Background results}
    \subsection{WMM$_0$}
        \lstinputlisting[basicstyle=\small\ttfamily]{03-04-WMM0-background.out}
        \lstinputlisting[basicstyle=\small\ttfamily]{00-WMM0-background-entropy.out}
        \begin{tikzpicture}
            \begin{axis}[ybar, bar width=2pt, enlargelimits=0,
                    xlabel=Distance to tail, ylabel=Motif hit count,
                    width=0.9\textwidth, height=200pt]
                \addplot[black] table[x=Distance, y=Count] {03-04-WMM0-background.tsv};
            \end{axis}
        \end{tikzpicture}

    \subsection{WMM$_1$}
        \lstinputlisting[basicstyle=\small\ttfamily]{03-04-WMM1-background.out}
        \lstinputlisting[basicstyle=\small\ttfamily]{00-WMM1-background-entropy.out}
        \begin{tikzpicture}
            \begin{axis}[ybar, bar width=2pt, enlargelimits=0,
                    xlabel=Distance to tail, ylabel=Motif hit count,
                    width=0.9\textwidth, height=200pt]
                \addplot[black] table[x=Distance, y=Count] {03-04-WMM1-background.tsv};
            \end{axis}
        \end{tikzpicture}

    \subsection{WMM$_2$}
        \lstinputlisting[basicstyle=\small\ttfamily]{03-04-WMM2-background.out}
        \lstinputlisting[basicstyle=\small\ttfamily]{00-WMM2-background-entropy.out}
        \begin{tikzpicture}
            \begin{axis}[ybar, bar width=2pt, enlargelimits=0,
                    xlabel=Distance to tail, ylabel=Motif hit count,
                    width=0.9\textwidth, height=200pt]
                \addplot[black] table[x=Distance, y=Count] {03-04-WMM2-background.tsv};
            \end{axis}
        \end{tikzpicture}

\section{Uniform background results}
    \subsection{WMM$_0$}
        \lstinputlisting[basicstyle=\small\ttfamily]{03-04-WMM0-uniform.out}
        \lstinputlisting[basicstyle=\small\ttfamily]{00-WMM0-uniform-entropy.out}
        \begin{tikzpicture}
            \begin{axis}[ybar, bar width=2pt, enlargelimits=0,
                    xlabel=Distance to tail, ylabel=Motif hit count,
                    width=0.9\textwidth, height=200pt]
                \addplot[black] table[x=Distance, y=Count] {03-04-WMM0-uniform.tsv};
            \end{axis}
        \end{tikzpicture}

    \subsection{WMM$_1$}
        \lstinputlisting[basicstyle=\small\ttfamily]{03-04-WMM1-uniform.out}
        \lstinputlisting[basicstyle=\small\ttfamily]{00-WMM1-uniform-entropy.out}
        \begin{tikzpicture}
            \begin{axis}[ybar, bar width=2pt, enlargelimits=0,
                    xlabel=Distance to tail, ylabel=Motif hit count,
                    width=0.9\textwidth, height=200pt]
                \addplot[black] table[x=Distance, y=Count] {03-04-WMM1-uniform.tsv};
            \end{axis}
        \end{tikzpicture}

    \subsection{WMM$_2$}
        \lstinputlisting[basicstyle=\small\ttfamily]{03-04-WMM2-uniform.out}
        \lstinputlisting[basicstyle=\small\ttfamily]{00-WMM2-uniform-entropy.out}
        \begin{tikzpicture}
            \begin{axis}[ybar, bar width=2pt, enlargelimits=0,
                    xlabel=Distance to tail, ylabel=Motif hit count,
                    width=0.9\textwidth, height=200pt]
                \addplot[black] table[x=Distance, y=Count] {03-04-WMM2-uniform.tsv};
            \end{axis}
        \end{tikzpicture}

\section{Conclusions}
    Seeing as how we are looking for the AATAAA pattern, 
        I would expect WMM$_0$ to perform the best.
    However, the histograms of motif hits between WMM$_0$ and WMM$_1$ are mostly indistinguishable.  
    WMM$_1$ had more spikes and didn't exclude as many candidates, but the shape of the plot seemed similar.
    
    On the other hand, WMM$_2$ showed fewer spikes, 
        but it's probability model also closely resembled that of the computed background model.
    The relative entropy of the MEME'ed model was also relatively low.  
    This is the result of using the computed background as the WMM (against a uniform background): \\
    \begin{tikzpicture}
        \begin{axis}[ybar, bar width=2pt, enlargelimits=0,
                xlabel=Distance to tail, ylabel=Motif hit count,
                width=0.9\textwidth, height=200pt]
            \addplot[black] table[x=Distance, y=Count] {03-04-background-uniform.tsv};
        \end{axis}
    \end{tikzpicture}
    
    Comparing the histograms, the spike at position 8 seems to be the result of bad filtering.
    Whereas the spike at 23-24 seems to be a consensus between the plots.  
    More interestingly, none of the WMM's beyhond the first one \textit{failed} to find a motif.  
    However, if I look at a few candidates, 
        none of their log-likelihoods are particularly high.
    From the limited sample I browsed, the maximums typically fell below 2.
    And the values were much lower for WMM$_1$ and WMM$_2$ compared to WMM$_0$.  

\end{document}