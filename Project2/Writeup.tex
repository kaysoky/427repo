\documentclass[a4paper, 12pt]{report}

\usepackage[margin=1in]{geometry}
\usepackage{amsmath}
\usepackage{amssymb}
\usepackage{listings}
\usepackage{pgfplots}
\pgfplotsset{compat=1.8}

\begin{document}
    \begin{center}
        {\LARGE Project Two} \\
        CSE 427 A \\
        Joseph Wu \makebox[0pt][l]{{\tiny (0911680)}} \\
        February 1-6, 2014 \\
        {\tiny Typeset via \LaTeX}
    \end{center}

\section{Build instructions}
    ``Histogram.out'' and ``Histogram.tex'' are generated by running: \\
    \texttt{python find\_ORF.py NC\_000909.fna NC\_000909.gbk --LaTeX > Histogram.out}

\section{Idea 1}
    Looking at the results from the simple scheme of matching stop positions, 
    we start seeing an equal or larger number of matches compared with non-matches 
        at a threshold of around 360 base pairs.
        
\section{Idea 2}
    The Markov chains start to match more than not around 159 base pairs,
        which is much better than the simplier heuristic.

\section{Histogram}
    The following is the ``histogram'' of open reading frames
    that end within the bounds of the annotated genes.
    It also includes the average log ratio,
        the number of ORFs with positive log ratios, 
        and the number of ORFs that fit both criterion of Ideas 1 and 2.

    I've also included a graphical histogram.
    \lstinputlisting[basicstyle=\scriptsize\ttfamily]{Histogram.out}
    
    \input{Histogram}
\end{document}